\newpage
%% BIBLIOGRAPHY AND OTHER LISTS

%% A small distance to the other stuff in the table of contents (toc)
\addtocontents{toc}{\protect\vspace*{\baselineskip}}

\raggedright
\sloppy

%% The Bibliography
\begin{thebibliography}{99}
\addcontentsline{toc}{section}{Literatura a~zdroje} %'Bibliography' into toc

\bibitem{hospitace-funkce} A. Culková. \emph{Hospitace a~její funkce: bakalářská práce}. Brno. 2008. Dostupné z~<\url{https://is.muni.cz/th/eqeju/Bakalarska_prace_-_AC.pdf}>. [Cit. 23. 2. 2025]
% obec pokec kompletní zdroj, hospitací to nekončí, ale začíná. Taky "hospitace v jídelně

\bibitem{ped-proces-rizeni} V. Trojan a~kol. \emph{Pedagogický proces a~jeho řízení}. Národní ústav pro vzdělávání. PedF UK Praha. 2012. 

\bibitem{ucime-ucit-se} H. Kasíková, V. Žák. \emph{Učíme děti učit se - hospitační arch}. Národní ústav pro vzdělávání. Praha. 2011. ISBN: 978-80-87063-50-7. Dostupné z~<\url{https://www.nuov.cz/uploads/AE/evaluacni_nastroje/12_Ucime_deti_ucit_se.pdf}>. [Cit. 23. 2. 2025]
%Nástroj pro učitele - kompetence k učení, třikrát opakovaný cyklus pozorování. Strukturované hodnocení (++ /-)

\bibitem{metody-a-formy} V. Žák. \emph{Metody a~formy výuky - hospitační arch}. Národní ústav pro vzdělávání. Praha. 2012. ISBN: 978-80-87063-61-3. Dostupné z~<\url{https://nuov.cz/uploads/AE/evaluacni_nastroje/11_Metody_a_formy_vyuky.pdf}> [Cit. 23. 2. 2025]
%Nástroj pro učitele - peer hospitace, pro zamyšlení, jestli výuka splňuje cíle a jestli k tomu zvolené metody fungují. Strukturované hodnocení (++ /-)

\bibitem{rozvijejici-hospotace} T. Janík. \emph{Rozvíjející hospitace aneb o~poznávání a~sdílení v~učitelské profesi}.  In: Setkání učitelů matematiky všech typů a~stupňů škol 2014. ISBN: 978-80-86843-45-2. Dostupné z~<\url{https://8d8f55af62.clvaw-cdnwnd.com/0023db53731df613e31376e312bef977/200000183-949cf949d1/srni2014.pdf}> [Cit. 23.2. 2025]
%nejsou nástrojem kontroly výkonu, ale podporují profesní rozvoj učitele: rozvíjející hospitace. Tím je odlišujeme od tzv. kontrolních hospitací
% rozvíjející hospitace - není nástroj kontroly, pomáhá analyzovat situaci ve vzdělávání

\bibitem{autoevaluace-zahranici} D. Vrabcová, L. Procházková, K. Rýdl. \emph{Autoevaluace školy v~zahraničí}. Národní ústav pro vzdělávání, školské poradenské zařízení a~zařízení pro další vzdělávání pedagogických pracovníků. Praha. 2012. ISBN 978-80-87063-75-0. Dostupné z~<\url{https://www.nuov.cz/uploads/AE/publikaceAutoevaluace_skoly_v_zahranici.pdf}> [Cit. 23. 2. 2025]


% \bibitem{hospitace-hodnocení} K. Holčáková. \emph{Hospitace a~její funkce, seminární práce}. Zlín. 2015. Dostupné z~\url{https://is.muni.cz/www/pacholik/Hospitace__hodnoceni_pracovniku_a_jejich_vykonu.pdf} [Cit. 23. 2. 2025]
% % pokec obec slabý

\bibitem{hospitace-prostredek} D. Krulišová. \emph{HOSPITACE JAKO PROSTŘEDEK ZVYŠOVÁNÍ KVALITY A~EVALUAČNÍ NÁSTROJ, bakalářská práce}. Praha. 2008. Dostupné z~<\url{https://dspace.cuni.cz/bitstream/handle/20.500.11956/15525/BPTX_2007_2_11410_OSZD001_81548_0_59417.pdf?sequence=1}> [Cit. 23. 2. 2025]

%Za největší úspěch hospitujícího je považováno, když sám učitel označí, co se mu v hodině povedlo a naopak, kde cítí své reservy. Určí slabá místa v hodině a doporučí způsob, jak je odstranit ke spokojenosti obou zúčastněných.

%hospitační arch - formální "odškrtávání" kolonek - bylo/nebylo v hodině přítomno 

\bibitem{nastroje-hodnoceni} V. Jakubovská a~kol. \emph{Nástroje hodnotenia vybraných kompetencií učiteľa}. Albert Boskovice. 2017. ISBN: 978-80-7326-283-9. 


\end{thebibliography}
