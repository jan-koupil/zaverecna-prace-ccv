%% Language %%%%%%%%%%%%%%%%%%%%%%%%%%%%%%%%%%%%%%%%%%%%%%%%%
\usepackage[czech]{babel}

\usepackage[T1]{fontenc}
\usepackage[utf8]{inputenc}
\usepackage{csquotes}

\usepackage{lmodern} %Type1-font for non-english texts and characters
										 %together with inputenc help pdf find ligatures like fi

\usepackage{enumitem}
\usepackage{tocloft}
\newcommand{\topic}{Význam hospitace v praxi ředitele školy}

\usepackage{a4wide} %%Smaller margins = more text per page.
\usepackage{fancyhdr} %%Fancy headings

%space betwwen paragraphs, keep the indent
\edef\restoreparindent{\parindent=\the\parindent\relax}


\usepackage{parskip}
%\restoreparindent

\usepackage[all]{nowidow}

% \usepackage[hyphens]{url}
% \usepackage{url}
% \UrlBreaks{\do\/\do-\do_}
\usepackage{pdfpages}

\usepackage[unicode,
%						 pdftex,
						pdfpagelabels,
            breaklinks
						]{hyperref}
\hypersetup{
            pdfauthor={Jan Koupil},
            pdftitle={\topic},
            %pdfsubject={Teaching and high school / university level experiments in the field of nuclear and particle physics},
            %pdfkeywords={nuclear physics, particle physics, radioactivity, high school teaching, university physics, school labs, school demonstrations},
						pdfstartpage=1,
%            pdfproducer={Latex with hyperref, or other system},
%            pdfcreator={pdflatex, or other tool}
						bookmarksnumbered,
						bookmarksopen=true,         
						bookmarksopenlevel=1, 
%						hidelinks
						colorlinks=true,				
						urlcolor=[RGB]{0 0 188},
						linkcolor=[RGB]{0 0 188},
						%citecolor=[RGB]{255 0 255}
						citecolor=[RGB]{0 0 188}
}
\usepackage{xurl}
\usepackage{hyperxmp}

						
\linespread{1.3}
%\usepackage{indentfirst}
\usepackage{verbatim}

\usepackage[style=footnote-dw, backend=biber, giveninits=false]{biblatex}



\addbibresource{inc/literatura.bib}

\renewbibmacro*{cite:title}{%
  \mkbibemph{\printfield{title}}%
}

\DeclareFieldFormat{title}{\mkbibemph{#1}}

\renewcommand*{\mkbibnamefamily}[1]{\textsc{#1}} % Příjmení kapitálkami
\renewcommand*{\mkbibnamegiven}[1]{#1}           % Křestní jména normálně

\renewcommand*{\multinamedelim}{. }
\renewcommand*{\finalnamedelim}{. }

\renewbibmacro*{cite:name}{%
  \def\multinamedelim{\addperiod\space}% Mezi více autory bude tečka a mezera
  \def\finalnamedelim{\addperiod\space}% Mezi posledním a předposledním autorem bude tečka a mezera
  \printnames[family-given]{author}%
  \addperiod
}



\newbibmacro*{footcite:isbn}{}

\renewbibmacro*{cite}{%
  \usebibmacro{cite:name}% Zobrazí autora
  \setunit{\addspace}%
  \printfield{title}% Zobrazí název
}
