%% Title Page %%%%%%%%%%%%%%%%%%%%%%%%%%%%%%%%%%%%%%%%%%%%%%%

%% The simple version:
%\hypersetup{pageanchor=false} 

% \author{Jan Koupil}
% \title{Význam hospitace v praxi ředitele školy}
% \date{Březen 2025}
% \maketitle

\includepdf[pages=1, pagecommand={}]{inc/title.pdf}

%\hypersetup{pageanchor=true} 
%% The nice version:
%\input{titlepage} %%You need a file 'titlepage.tex' for this.
%% ==> TeXnicCenter supplies a possible titlepage file
%% ==> with its templates (File | New from Template...).

\newpage

\vspace*{\fill}

\noindent \textbf{Čestné prohlášení:} \\

Prohlašuji, že jsem závěrečnou práci na téma \textbf{\topic}
vypracoval/a samostatně a na základě literatury a pramenů uvedených v Seznamu použité literatury a moje závěrečná práce splňuje podmínky uvedené v § 31 zákona č.121/2000 Sb., autorského zákona.

\vspace{1cm}

\hfill V Pardubicích dne \dotfill  
\vspace{1cm}

\hfill
Jan Koupil

\newpage

\vspace*{\fill}

\noindent \textbf{OPRÁVNĚNÍ K VÝKONU PRÁVA VYUŽÍT DÍLO} \\

Zpracovaná závěrečná práce na základě tohoto oprávnění má povahu autorského díla a účastník (autor) dává organizátorovi kurzu (CCV Pardubice) právo dílo použít dle § 12 Zákona 121/2000 Sb. Autorský zákon, zejména právo na vystavování originálu, tj. použít díla jako ukázku zpracované závěrečné práce účastníkům kvalifikačních kurzů.

\vspace{1cm}

\hfill V Pardubiích dne \dotfill  

\hfill
Jan Koupil

\newpage
