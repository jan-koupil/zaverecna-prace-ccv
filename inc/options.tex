%% Language %%%%%%%%%%%%%%%%%%%%%%%%%%%%%%%%%%%%%%%%%%%%%%%%%
\usepackage[czech]{babel} %francais, polish, spanish, ...
% %\usepackage[T1]{fontenc}
% %\usepackage[ansinew]{inputenc}
% \usepackage[utf8]{inputenc}


\usepackage[T1]{fontenc}
\usepackage[utf8]{inputenc}
\usepackage{csquotes}

\usepackage{lmodern} %Type1-font for non-english texts and characters
										 %together with inputenc help pdf find ligatures like fi

%% Packages for Graphics & Figures %%%%%%%%%%%%%%%%%%%%%%%%%%
%%\usepackage{graphicx} %%For loading graphic files
\usepackage{graphbox} %loads graphicx package
\graphicspath{{img/}}

%\usepackage{luavlna} 
% \usepackage{pbox}

%\usepackage{subfig} %%Subfigures inside a figure
\usepackage{caption,subcaption} %%Subfigures inside a figure
%\usepackage{pst-all} %%PSTricks - not useable with pdfLaTeX
%\usepackage[titletoc,title]{appendix}


\usepackage{tocloft}
%\renewcommand{\cftchapleader}{\cftdotfill{\cftdotsep}} % leading dots for chapters


%% Please note:
%% Images can be included using \includegraphics{Dateiname}
%% resp. using the dialog in the Insert menu.
%% 
%% The mode "LaTeX => PDF" allows the following formats:
%%   .jpg  .png  .pdf  .mps
%% 
%% The modes "LaTeX => DVI", "LaTeX => PS" und "LaTeX => PS => PDF"
%% allow the following formats:
%%   .eps  .ps  .bmp  .pict  .pntg

\newcommand{\topic}{Význam hospitace v praxi ředitele školy}


%% Math Packages %%%%%%%%%%%%%%%%%%%%%%%%%%%%%%%%%%%%%%%%%%%%
% \usepackage{amsmath}
% \usepackage{amsthm}
% \usepackage{amsfonts}
% \usepackage{amssymb}
%% Line Spacing %%%%%%%%%%%%%%%%%%%%%%%%%%%%%%%%%%%%%%%%%%%%%
% \usepackage{setspace}
%\singlespacing        %% 1-spacing (default)
% \onehalfspacing       %% 1,5-spacing
%\doublespacing        %% 2-spacing

%% Other Packages %%%%%%%%%%%%%%%%%%%%%%%%%%%%%%%%%%%%%%%%%%%
\usepackage{a4wide} %%Smaller margins = more text per page.
\usepackage{fancyhdr} %%Fancy headings
% \usepackage{longtable} %%For tables, that exceed one page
% \usepackage{upgreek} %%non-italic Greek like \upalpha

%\usepackage[usenames,dvipsnames]{xcolor}
%\colorlet{myurlcolor}{Aquamarine}

%space betwwen paragraphs, keep the indent
\edef\restoreparindent{\parindent=\the\parindent\relax}

%too many unprocessed floats
%\extrafloats{100}

\usepackage{parskip}
%\restoreparindent

\usepackage[all]{nowidow}

%%%%%%%%%%%%%%%%%%%%%%%%%%%%%%%%%%%%%%%%%%%%%%%%%%%%%%%%%%%%%
%% OPTIONS
%%%%%%%%%%%%%%%%%%%%%%%%%%%%%%%%%%%%%%%%%%%%%%%%%%%%%%%%%%%%%
%%
%% ATTENTION: You need a main file to use this one here.
%%            Use the command "\input{filename}" in your
%%            main file to include this file.
%%
%\usepackage[unicode,pdfpagelabels,hypertex]{hyperref}
\usepackage[hyphens]{url}
\usepackage{pdfpages}

\usepackage[unicode,
%						 pdftex,
						pdfpagelabels%,
						]{hyperref}
\hypersetup{
            pdfauthor={Jan Koupil},
            pdftitle={\topic},
            %pdfsubject={Teaching and high school / university level experiments in the field of nuclear and particle physics},
            %pdfkeywords={nuclear physics, particle physics, radioactivity, high school teaching, university physics, school labs, school demonstrations},
						pdfstartpage=1,
%            pdfproducer={Latex with hyperref, or other system},
%            pdfcreator={pdflatex, or other tool}
						bookmarksnumbered,
						bookmarksopen=true,         
						bookmarksopenlevel=1, 
%						hidelinks
						colorlinks=true,				
						urlcolor=[RGB]{0 0 188},
						linkcolor=[RGB]{0 0 188},
						%citecolor=[RGB]{255 0 255}
						citecolor=[RGB]{0 0 188}
}
% \usepackage[hyphenbreaks]{breakurl}
\usepackage{hyperref}

\usepackage{hyperxmp}

%\usepackage[pdftex]{hyperref}
						
\linespread{1.3}
%\usepackage{indentfirst}
\usepackage{verbatim}
% \usepackage{ifthen}
% %\usepackage{commath} //kvuli derivacim, resi physics

\usepackage{url}

\usepackage[style=footnote-dw, backend=biber, giveninits=false]{biblatex}


% \DefineBibliographyStrings{czech}{%
%   bibliography = {Zdroje a literatura}
% }

% \renewcommand{\refname}{Zdroje a literatura} % Pro články (article class)


\addbibresource{inc/literatura.bib}

\renewbibmacro*{cite:title}{%
  \mkbibemph{\printfield{title}}%
}

\DeclareFieldFormat{title}{\mkbibemph{#1}}

\renewcommand*{\mkbibnamefamily}[1]{\textsc{#1}} % Příjmení kapitálkami
\renewcommand*{\mkbibnamegiven}[1]{#1}           % Křestní jména normálně

\renewcommand*{\multinamedelim}{. }
\renewcommand*{\finalnamedelim}{. }

\renewbibmacro*{cite:name}{%
  \def\multinamedelim{\addperiod\space}% Mezi více autory bude tečka a mezera
  \def\finalnamedelim{\addperiod\space}% Mezi posledním a předposledním autorem bude tečka a mezera
  \printnames[family-given]{author}%
  \addperiod
}



% \renewbibmacro*{finentry}{%
%   \printfield{labelprefix}%
%   \printfield{usera}%
%   \printfield{userb}%
%   \printfield{userc}%
% %   \printfield{doi}%
%   \printfield{isbn}%
% %   \printlist{location}% Zobrazení místa vydání
% %   \setunit{\addcomma\space}%
% %   \printfield{year}% Zobrazení roku
% %   \setunit{\addperiod\space}%
% %   \printfield{note}% Zobrazení pole "note" až na konci
%   \finentry}



% \renewbibmacro*{footcite:note}{}
\newbibmacro*{footcite:isbn}{}

\renewbibmacro*{cite}{%
  \usebibmacro{cite:name}% Zobrazí autora
  \setunit{\addspace}%
  \printfield{title}% Zobrazí název
  % \setunit{\addperiod\space}%
  % \iffieldundef{pages}{}{%
  %   \addspace~\lowercase{s.}~\printfield{pages} % Explicitně vynutíme malé "s."
  % }
  % \finentry
}

% \renewbibmacro*{bib:entry}{%
%   \usebibmacro{bib:author}% Autor
%   \setunit{\addspace}%
%   \usebibmacro{bib:title}% Název
%   \setunit{\addspace}%
%   \printlist{publisher}% Nakladatel
%   \setunit{\addcomma\space}%
%   \printlist{location}% Město
%   \setunit{\addcomma\space}%
%   \printdate% Rok
%   \setunit{\addperiod\space}%
%   \printfield{note}% Note se zobrazí až nakonec
%   \finentry
% }



% \usepackage{siunitx} %jednotky SI
% \sisetup{locale = US}
% \DeclareSIUnit\clight{\textit{c}}
% \DeclareSIUnit\astronomicalunit{AU}
% \DeclareSIUnit\year{yr}


% \usepackage{physics}

% \usepackage[version=4]{mhchem} %kvůli \ce - chemickým vzorcům
% \usepackage{array}
% \usepackage{tabu} %tabulky s mezerami a centrováním

% %\usepackage[hypcap]{caption} %v pdf vede klik na obr nahoru, ne na label
% \usepackage[hypcap,format=plain,font=it]{caption}

% %\usepackage[resetfonts]{cmap} %ligatures can by found by search in PDF - nefachá
% \usepackage{enumitem}
% \usepackage{booktabs}
% %\ifthenelse{ \equal{\pdf}{1} } {\usepackage{pdfpages}}{}
% \usepackage{pdflscape}

% %%%%%%%%%%%%%%%%%%%%%%%%%%%%%%%%%%%%%%%%%%%%%%%%%%%%%%%%%%%%%
% %% Abbreviations
% %%%%%%%%%%%%%%%%%%%%%%%%%%%%%%%%%%%%%%%%%%%%%%%%%%%%%%%%%%%%%

% %%Definitions for Figure-References
% %\newcommand{\fig}[1]{(figure \ref{#1})}
% %\newcommand{\FIG}[1]{figure \ref{#1}}

% %%Definitions for Table-References
% %\newcommand{\tref}[1]{table~\ref{#1}}
% %\newcommand{\Tref}[1]{Table~\ref{#1}}

% %%Definitions for Page-References
% \newcommand{\page}[1]{(page \pageref{#1})}
% %\newcommand{\PAGE}[1]{page \pageref{#1}}

% %%Definitions for Code
% %\newcommand{\precode}[1]{\textbf{\footnotesize #1}}
% %\newcommand{\code}[1]{\texttt{\footnotesize #1}}

% %%Defines and sets length in one command
% \newcommand{\deflen}[2]{%      
%     \expandafter\newlength\csname #1\endcsname
%     \expandafter\setlength\csname #1\endcsname{#2}%
% }
% \deflen{prtscr}{7cm}
% \deflen{hist}{14.2cm}
% \deflen{tableskip}{-3ex}

% \newcommand{\am}{\elm{Am}{241}}
% \newcommand{\keV}{\kilo\electronvolt}
% \newcommand{\MeV}{\mega\electronvolt}
% \newcommand{\cm}{\centi\meter}
% \newcommand{\mm}{\milli\meter}
% \newcommand{\fref}[1]{\hyperref[#1]{Fig.~\ref{#1}}}
% \newcommand{\Fref}[1]{\hyperref[#1]{Fig.~\ref{#1}}}
% \newcommand{\tref}[1]{\hyperref[#1]{Tab.~\ref{#1}}}
% \newcommand{\Tref}[1]{\hyperref[#1]{Tab.~\ref{#1}}}
% \newcommand{\subfref}[2]{\hyperref[#2]{Fig.~\ref{#1}\,\subref{#2}}}
% \newcommand{\subFref}[2]{\hyperref[#2]{Fig.~\ref{#1}\,\subref{#2}}}
% \newcommand{\expref}[1]{\hyperref[#1]{experiment \ref{#1}}}
% \newcommand{\labref}[1]{\hyperref[#1]{lab \ref{#1}}}
% %\renewcommand{\eqref}[1]{\hyperref[#1]{equation (\ref{#1})}}
% \newcommand{\formularef}[1]{\hyperref[#1]{formula \eqref{#1}}}

% %\newcommand{\prg}[1]{\textsf{{#1}}}
% \newcommand{\prg}[1]{#1}
% %\newcommand{\eqp}[1]{\noindent\textbf{\emph{Equipment:}} #1}
% %\newcommand{\steps}[1]{\noindent\textbf{\emph{Instructions:}} #1}
% \newcommand{\steps}[1]{\noindent\emph{Instructions:} #1}

% \newcommand{\menu}[1]{\emph{#1}} %for locations in menu etc.
% \newcommand{\rarr}[0]{\ensuremath{\rightarrow}}
% \newcommand{\bigcell}[2]{\begin{tabular}{@{}#1@{}}#2\end{tabular}}
% \newcommand{\note}[1]{\noindent\emph{Note}: #1}


% \newenvironment{xpsetup}
% {
% 		\newcommand{\xptype}[1]{Experiment type: & \multicolumn{4}{p{0.77\textwidth}}{##1} \\}
% 		\newcommand{\xpdemo}{\xptype{Demonstration}}
% 		\newcommand{\xplab}{\xptype{Laboratory}}
	
% 		\newcommand{\xpdur}[1]{Duration: & \multicolumn{4}{p{0.77\textwidth}}{##1}\\}
% 		\newcommand{\xpeqp}[1]{Equipment: & \multicolumn{4}{p{0.77\textwidth}}{##1} \\}
		
% 		\newcommand{\xpsettings}{Settings:}
		
% 		\newcommand{\xpsrc}[1]{Radiation source & ##1}
% 		\newcommand{\xpsrcam}{\xpsrc{241~Am}}
% 		\newcommand{\xpsrcu}{\xpsrc{Uranium glass}}
% 		\newcommand{\xpsrcth}{\xpsrc{Thorium electrode}}
% 		\newcommand{\xpsrcother}{\xpsrc{Other}}
		
% 		\newcommand{\xpmode}[1]{Mode & ##1}
% 		\newcommand{\xpmodspect}{\xpmode{Spectrometer}}
% 		\newcommand{\xpmodcount}{\xpmode{Counter}}
		
% 		\newcommand{\xpanatype}[1]{Analysis type & ##1}
% 		\newcommand{\xpanabasic}{\xpanatype{Basic}}
% 		\newcommand{\xpanaext}{\xpanatype{Extend}}
% 		\newcommand{\xpanaoff}{\xpanatype{Off}}
		
% 		\newcommand{\xpbias}[1]{Bias voltage & \SI{##1}{\volt}}
		
% 		\newcommand{\xpyes}{Yes}
% 		\newcommand{\xpno}{No}
		
% 		\newcommand{\xpcontinuous}[1]{Continuous m. & ##1}
% 		\newcommand{\xpcontyes}{\xpcontinuous{\xpyes}}
% 		\newcommand{\xpcontno}{\xpcontinuous{\xpno}}
		
% 		\newcommand{\xpintegral}[1]{Integral mode & ##1}
% 		\newcommand{\xpintyes}{\xpintegral{\xpyes}}
% 		\newcommand{\xpintno}{\xpintegral{\xpno}}
		
% 		\newcommand{\xpcnt}[1]{Exp.~count & ##1}
% 		\newcommand{\xptime}[1]{Exp.~time & ##1}
		
% 		\newcommand{\xpminlvl}[1]{Min.~level & ##1}
% 		\newcommand{\xpmaxlvl}[1]{Max.~level & ##1}
		
% 		\newcommand{\xpcolor}[1]{Colormap & ##1}
% 		\newcommand{\xpcolhot}{\xpcolor{Hot}}		
% 		\newcommand{\xpcolgray}{\xpcolor{Gray}}		
	
% 		\vspace{0.5\baselineskip}

% 		\noindent\begin{tabular}{@{}>{\em}p{0.18\textwidth} >{\em}l<{:} p{0.20\textwidth} >{\em}l<{:} p{0.18\textwidth}}
% 		\toprule	
% }
% {		
% 		\bottomrule
% 		\end{tabular}
% }




%remove the ``chapter'' text from chapter titles
% \usepackage{titlesec}
% \titleformat{\chapter}[block]
%   {\normalfont\huge\bfseries}{\thechapter.}{1em}{\huge}
% \titlespacing*{\chapter}{0pt}{-19pt}{0pt}
%\usepackage{titlesec}
%\titleformat{\chapter}[hang] 
%{\normalfont\huge\bfseries}{\chaptertitlename\ \thechapter:}{1em}{} 

% přepneme indexy v matematice na neskloněné písmo
% \catcode`_=\active
% \newcommand_[1]{\ensuremath{\sb{\mathrm{#1}}}}

\usepackage{color}


% \newcommand{\elm}[2]{\ce{^#2}{#1}}

% \newenvironment{tightcenter}{%
%   \setlength\topsep{0pt}
%   \setlength\parskip{0pt}
%   \begin{center}
% }{%
%   \end{center}
% }

% %hms macro - for time formatting
% %http://tex.stackexchange.com/questions/38905/time-of-the-day-or-time-period-using-the-package-siunitx
% \ExplSyntaxOn
% \NewDocumentCommand \hms { o > { \SplitArgument { 2 } { ; } } m }
%   {
%     \group_begin:
%       \IfNoValueF {#1}
%         { \keys_set:nn { siunitx } {#1} }
%       \siunitx_hms_output:nnn #2
%     \group_end:
%   }
% \cs_new_protected:Npn \siunitx_hms_output:nnn #1#2#3
%   {
%     \IfNoValueF {#1}
%       {
%         \tl_if_blank:nF {#1}
%           {
%             \SI {#1} { \hour }
%             \IfNoValueF {#2} { ~ }
%           }
%       }
%     \IfNoValueF {#2}
%       {
%         \tl_if_blank:nF {#2}
%           {
%             \SI {#2} { \minute }
%             \IfNoValueF {#3} { ~ }
%           }
%       }
%     \IfNoValueF {#3}
%       { \tl_if_blank:nF {#3} { \SI {#3} { \second } } }
%   }
% \ExplSyntaxOff

%\usepackage{etoolbox}% http://ctan.org/pkg/etoolbox
%\makeatletter %avoid hyphenation in toc
%% \patchcmd{<cmd>}{<search>}{<replace>}{<success>}{<failure>}
%\patchcmd{\@makechapterhead}{#1}{\hyphenpenalty=10000 #1}{}{}% Patch \chapter
%\patchcmd{\@makeschapterhead}{#1}{\hyphenpenalty=10000 #1}{}{}% Patch \chapter*
%\patchcmd{\@makesectionhead}{#1}{\hyphenpenalty=10000 #1}{}{}% Patch \section
%\patchcmd{\@makessectionhead}{#1}{\hyphenpenalty=10000 #1}{}{}% Patch \section*
%\patchcmd{\@makesubsectionhead}{#1}{\hyphenpenalty=10000 #1}{}{}% Patch \subsection
%\patchcmd{\@makessubsectionhead}{#1}{\hyphenpenalty=10000 #1}{}{}% Patch \subsection*
%\makeatother